\section{Техническое задание}
\subsection{Исходные данные}
	Список стран, максимум 11000 записей. В файле каждая страна должна быть разделена переносом строки, поля разделены пробелом. Информация о стране: название, столица, материк, необходимость наличия визы, время полета до страны, минимальная стоимость отдыха, основной вид туризма:
	\begin{enumerate}
		\item Экскурсионный: 
		\begin{enumerate}
			\item Количество объектов
			\item Основной вид объектов (природа, искусство, история)
		\end{enumerate}
		\item Пляжный:
		\begin{enumerate}
			\item Основной сезон
			\item Температура воздуха и воды
		\end{enumerate}
		\item Спортивный:
		\begin{enumerate}
			\item Вид спорта (горные лыжи, серфинг, восхождения)
		\end{enumerate}
	\end{enumerate}

\noindent\textbf{Меню программы:}
\begin{enumerate}
	\item Вывести информацию о программе
	\item Загрузить список стран из файла
	\item Сохранить список стран в файл
	\item Добавить страну в конец списка
	\item Удалить страну из списка по названию
	\item Вывести список стран
	\item Вывести список ключей
	\item Вывести список стран по по списку ключей
	\item Вывести список стран на выбранном материке, где можно заняться указанным видом спорта, со стоимостью отдыха меньше указанной
	\item Отсортировать список ключей по столице
	\item Отсортировать список фильмов по столице
	\item Произвести и вывести исследование
	\item Выход
\end{enumerate}


\subsection{Описание задачи, реализуемой программой}
	Сохранение, добавление, удаление, вывод, сортировка стран и/или ключей как в файл, так и в консоль. Проведение замерного эксперимента и исследования об эффективности использования массива структур и массива ключей

\subsection{Способ обращения к программе}
	Запускается через терминал: \texttt{./app.exe}. Затем необходимо выбрать одну из предложенных опций в меню. При выборе опции №12 (Произвести и вывести исследование) для построение графиков зависимости времени выполнения сортировки от количества элементов необходимо из корневой директории прописать команду: \texttt{gnuplot research/src/*.sh}. После этого графики будут расположены в директории \texttt{research/plots}

\subsection{Описание аварийных случаев}
\begin{enumerate}
	\item\textbf{Аварийные случаи при записи и чтении фильма}
	\begin{enumerate}
		\item Ошибки при вводе любого строкового поля
		\begin{enumerate}
			\item Пустой ввод 
			\item Переполнение буфера
		\end{enumerate}
		\item Ошибки при вводе поля: Информации о необходимости наличия визы
		\begin{enumerate}
			\item Пустой ввод 
			\item Символ
			\item Число, отличное от 0 или 1
		\end{enumerate}
		\item Ошибки при вводе полей: Время полета до страны, Минимальная стоимость отдыха, Количество объектов
		\begin{enumerate}
			\item Пустой ввод 
			\item Символ
			\item Отрицательное число
		\end{enumerate}
		\item Ошибки при вводе полей: Основной вид объектов, Вид спорта:
		\begin{enumerate}
			\item Пустой ввод 
			\item Символ
			\item Число не из диапазона [1; 3];
		\end{enumerate}
		\item Ошибки при вводе полей: Температура воздуха, Температура воды:
		\begin{enumerate}
			\item Пустой ввод 
			\item Символ
			\item Число, выходящее за границы типа \texttt{short}
		\end{enumerate}
	\end{enumerate}
	
	%\vspace{0.5em}
	\item\textbf{Аварийный случай при добавлении страны в конец списка}
	\begin{enumerate}
		\item Список уже содержит максимальное количество элементов
	\end{enumerate}
	
	\item\textbf{Аварийный случай при удалении страны из списка}
	\begin{enumerate}
		\item Страна отсутствует в списке
	\end{enumerate}
	
	\item\textbf{Аварийный случай при работе с файлом}
	\begin{enumerate}
		\item Файл отсутствует
		\item Нет прав доступа для работы с файлом
	\end{enumerate}
\end{enumerate}
