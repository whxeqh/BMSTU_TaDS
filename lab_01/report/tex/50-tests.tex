\chapter{Тестовые данные}
{\noindent \textbf {Позитивные тесты}}
\begin{table}[ht]
	\centering
	\resizebox{\textwidth}{!}{%
		\begin{tabular}{|c|c|c|c|}
			\hline
			\textbf{Тест}& \textbf{Целое число} & \textbf{Вещественное число} & \textbf{Вывод} \\
			\hline
			Числа равны, обычная запись & 123 & 123 & +0.1E1 \\
			\hline
			\makecell{Числа равны\\экспоненциальная запись} & 123 & 1.23e2 & +0.1E1 \\
			\hline
			Есть ведущие нули & 005 & 008.0 & +0.625E0 \\
			\hline
			Делимое равно 0 & 0 & 123 & +0.0E+1 \\
			\hline
			Делитель равен 0 & 123 & 0 & Деление 0 \\
			\hline
			Делимое больше делителя & 12345678 & 12.3e3 & \makecell{+0.100371365\\853658536585\\365853658536\\585365853E+4} \\
			\hline
			Делитель больше делимого & 123 & 4242.24e76 & \makecell{+0.289941163\\159085766010\\409594930979\\859696763E-77} \\
			\hline
			Делимое кратно делителю & 144 & 12e0 & +0.12E+2 \\
			\hline
			\makecell{Бесконечная дробь\\есть округления} & 5 &  0.0003e-1 & \makecell{+0.166666666\\666666666666\\666666666666\\6666667E+6}\\
			\hline
			\makecell{Бесконечная дробь\\нет округления} & 5 & .2e-3 &  \makecell{+0.333333333\\333333333333\\333333333333\\3333333E+668} \\
			\hline
			40 цифр в делителе & 31314 & \makecell{123456789012\\345678901234\\567890123456\\7890e31} & \makecell{+0.253643402\\282790620773\\394649037891\\3062448E-30} \\
			\hline
			40 цифр в делимом & \makecell{1234567890\\1234567890\\12345678901\\234567890e31} & 4224e-3124 & \makecell{+0.292274595\\199682004974\\513655042905\\9109588E+3160} \\
			\hline
			40 цифр в делителе и делимом & \makecell{1234567890\\1234567890\\1234567890\\1234567890} & \makecell{1111111111\\1111111111\\1111111111\\1111111111} & \makecell{+0.1111111\\1011111111\\1011111111\\10111111111\\01E+2} \\
			\hline
			Машинный бесконечность & 123 & 0.000001e-99999 &\makecell{Достигнута машинная\\бесконечность!}\\
			\hline
			Машинный ноль & 123 & 1e99999 & \makecell{Достигнут машинный\\ноль }\\
			\hline
		\end{tabular}
	} % Закрываем \resizebox
\end{table}

\newpage
{\noindent \textbf {Негативные тесты}}
\begin{table}[ht]
	\centering
	\resizebox{\textwidth}{!}{%
		\begin{tabular}{|c|c|c|c|}
			\hline
			\textbf{Тест}& \textbf{Целое число} & \textbf{Вещественное число} & \textbf{Вывод} \\
			\hline
			\makecell{Нет значащих цифр в\\вещественном числе} & 123 & . & \makecell{Ошибка при вводе\\действительного\\числа}\\
			\hline
			Две точки в вещественном числе & 123 & 1.1.1 & \makecell{Ошибка при вводе\\действительного\\числа}\\
			\hline
			Два знака +- в вещественном числе & 123 & +-23e3 & \makecell{Ошибка при вводе\\действительного\\числа}\\
			\hline
			Нет цифры после экспоненты & 123 & 12.3e &\makecell{Ошибка при вводе\\действительного\\числа}\\
			\hline
			Нет цифры перед экспонентой & 123 & e3 & \makecell{Ошибка при вводе\\действительного\\числа}\\
			\hline
			Символ перед экспонентой & 123 & 12ke3 & \makecell{Ошибка при вводе\\действительного\\числа}\\
			\hline
			Символ после экспоненты & 123 & 12.3ek & \makecell{Ошибка при вводе\\действительного\\числа}\\
			\hline
			Две экспоненты & 123 & 12.3e3e5 &\makecell{Ошибка при вводе\\действительного\\числа}\\
			\hline
			Два знака +- после экспоненты &  & 12.3e+-3 & \makecell{Ошибка при вводе\\действительного\\числа}\\
			\hline
			Пустой ввод &  & 12ke3 &\makecell{Ошибка при вводе\\целого числа}\\
			\hline
			Символ в числе & 12k3 & 12.3ek &\makecell{Ошибка при вводе\\целого числа}\\
			\hline
			Ввод вещественного числа & 123.12 & 12.3e3e5 & \makecell{Ошибка при вводе\\целого числа}\\
			\hline
		\end{tabular}
	} % Закрываем \resizebox
\end{table}

