\section{Описание структуры данных}

\begin{lstlisting}[caption={Структура для хранения фильма с вариативным полем}, label={lst:listing1}]
typedef struct 
{
	char name[MAX_COUNTRY_NAME_LENGTH];
	char capital[MAX_CAPITAL_LENGTH];       // Название столицы
	char mainland[MAX_MAINLAND_LENGTH];     // Название материка
	bool visa;                              // Потребность в визу
	
	uint32_t flying_time;                   // Время полета в минутах
	uint32_t min_vacation_price;            // Минимальная цена отдыха
	
	type_of_tourism tourism;                // Перечисление туризма
	type_t type;                            // Объекдинение структур туризма
} country_t;
\end{lstlisting}

\noindent\textbf{Объяснение полей:}
\newline
\begin{tabular}{|l|l|}
	\hline
	\textbf{Поле} & \textbf{Описание} \\
	\hline
	\texttt{name} & \makecell{Название страны} \\
	\hline
	\texttt{capital} & \makecell{Название столицы} \\
	\hline
	\texttt{mainland} & \makecell{Название материка} \\
	\hline
	\texttt{visa} & \makecell{потребность в визе}\\%антиссе }\\
	\hline
	\texttt{lfying\_time} & \makecell{Время полета} \\
	\hline
	\texttt{min\_vacation\_price} & \makecell{Минимальная стоимость\\отдыха} \\
	\hline
	\texttt{tourism} & \makecell{Вид туризма} \\
	\hline
	\texttt{type} & \makecell{Union туризма}\\%антиссе }\\
	\hline
\end{tabular}
\newpage

\begin{lstlisting}[caption={Объединение видов туризма}, label={lst:listing2}]
typedef union 
{
	sightseeing_t sightseeing;              // Экскурсионный
	beach_t beach;                          // Пляжный
	sport_t sport;                          // Спортивный
} type_t;
\end{lstlisting}

\noindent\textbf{Объяснение полей:}
\newline
\begin{tabular}{|l|l|}
	\hline
	\textbf{Поле} & \textbf{Описание} \\
	\hline
	\texttt{sightseeing} & \makecell{Структура с экскурсионным\\видом туризма} \\
	\hline
	\texttt{beach} & \makecell{Структура с пляжным\\видом туризма} \\
	\hline
	\texttt{sport} & \makecell{Структура со спортивным\\видом туризма} \\
	\hline
\end{tabular}
\vspace{2em}

\begin{lstlisting}[caption={Структуры туризма}, label={lst:listing3}]
	// Экскурсионный вид отдыха
	typedef struct
	{
		uint32_t objects_amount;           // Количество объектов
		type_of_objects objects_type;      // Вид объекта
	} sightseeing_t;   
	
	// Пляжный вид отдыха
	typedef struct
	{
		char season[MAX_SEASON_LENGTH];      // Сезон
		short water_temperature;             // Темеература воды
		short air_temperature;               // Температура воздуха
	} beach_t;
	
	// Спорттивный вид отдыха
	typedef struct
	{
		type_of_sport sport_type;
	} sport_t;
\end{lstlisting}

\noindent\textbf{Объяснение полей:}
\newline
\begin{tabular}{|l|l|}
	\hline
	\textbf{Поле} & \textbf{Описание} \\
	\hline
	\texttt{objects\_amount} & Количество объектов \\
	\hline
	\texttt{objects\_type} & Вид объекта \\
	\hline
	\texttt{season} & Сезон \\
	\hline
	\texttt{water\_temperature} & Температура воды \\
	\hline
	\texttt{air\_temperature} & Температура воздуха \\
	\hline
	\texttt{sport\_type} & Вид спорта \\
	\hline
\end{tabular}
\vspace{2em}
\begin{lstlisting}[caption={Возможные поля туризма}, label={lst:listing4}]
	//Вид туризма
	typedef enum 
	{
		SIGHTSEEING = 1,      // Экскурсионный
		BEACH,                // Пляжный
		SPORT                 // Спортивный 
	} type_of_tourism;
	
	//Вид объектов
	typedef enum 
	{
		NATURE = 1,            // Природа
		ART,                   // Искусство
		HISTORY                // История 
	} type_of_objects;
	
	//Вид спорта
	typedef enum 
	{
		MOUNTAIN_SKIING = 1,    // Горные лыжи
		SURFING,                // Сёрфинг
		CLIMBING                //Скалолазание
	} type_of_sport;
\end{lstlisting}