\chapter{Техническое задание}
\section{Исходные данные}
	На вход программе подаются два числа, целое и действительное, каждое из которых записано на отдельной строке.\par 
	Формат целого числа: [+-][0-9]+. Количество значащих цифр не более 40. \par
	Формат вещественного числа: [+-]?(0.[0-9]*)E[+-]?([0-9]*). Количество значащих цифр не более 40, количество цифр в порядке не более 5\par
	При корректно введенных данных и существовании ответа будет выведено число в формате: [+-]?(0.[0-9]*)E[+-]?([0-9]*). Количество значащих цифр не более 40, количество цифр в порядке не более 5\par
	При некорректно введенных данных будет выведено сообщение об ошибке. При слишком большом числе в ответе будет выведено <<Достигнута машинная бесконечность>> При слишком маленьком числе в ответе будет выведено: <<Достигнут машинный ноль>>


\section{Описание задачи, реализуемой программой}
	Деление целого числа на действительное число

\section{Способ обращения к программе}
	Запускается через терминал. Сначала вводится целое число, потом вещественное

\section{Описание аварийных случаев}
	\textbf{Аварийные случаи при вводе целого числа}
	\begin{enumerate}
		\item Пустой ввод
		\item Наличие посторонних символов
		\item Наличие более одного знака +- или неправильное его положение
		\item Количество значащих цифр больше 40
	\end{enumerate}
	\newpage
	\textbf{Аварийные случаи при вводе вещественного числа}
	\begin{enumerate}
		\item Отсутствие значащих цифр
		\item Отсутствие цифр в экспоненте
		\item Наличие посторонних символов 
		\item Наличие двух точек или двух экспонент
		\item Количество значащих цифр больше 40
		\item Количество цифр в экспоненте больше 5
	\end{enumerate}


