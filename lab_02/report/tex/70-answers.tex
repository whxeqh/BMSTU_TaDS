\section{Ответы на вопросы}
\noindent\textbf{1. Как выделяется память под вариантную часть записи?}\par
Для вариантной части записи используется объединение. Где каждое поле объединения - структура возможной записи. Объём такого объединения равен максимальному объёму одной из его структур. Такой формат обеспечивает минимизацию использования памяти.\newline

\noindent\textbf{2. Что будет, если в вариантную часть ввести данные, несоответствующие
описанным?}\par
Если такой ввод не вернет ошибку в программе, то дальнейший исход событий не предсказуем. Могут произойти разные ошибки, в том числе аварийное завершение программы. \newline

\noindent\textbf{3. Кто должен следить за правильностью выполнения операций с вариантной}\par
Только программист. Он должен обеспечить правильную запись вариантной части или вывод ошибки, при неверном вводе от пользователя.\newline

\noindent\textbf{4. Что представляет собой таблица ключей, зачем она нужна?}\par
Таблица ключей - структура данных, содержащая ключи или идентификаторы к исходной таблице. Таблица ключей занимает меньше памяти, поэтому выгодна в при частом использовании исходной таблицы.\newline

\noindent\textbf{5. В каких случаях эффективнее обрабатывать данные в самой таблице, а когда –
	использовать таблицу ключей?}\par
Если при работе с исходной таблицей часто происходит обмен, сравнение, вставка, удаление объектов, то использование таблицы ключей поможет сократить время работы этих подпрограмм
Если происходит обращение к большому количеству разных полей таблицы, то обрабатывать данные в таблице выгоднее\newline

\noindent\textbf{6. Какие способы сортировки предпочтительнее для обработки таблиц и почему?}\par
Если данные в таблице почти отсортированы, то сортировка вставками подойдет лучше всего, так как будет осуществлено небольшое количество обменов
\newline
В общем случае, очевидно, лучше всего работают сортировки с временной сложностью $O(N \log N)$. Так как быстрее этой константы отсортировать физически невозможно. Однако есть всеми любимая \textbf{Сталинская сортировка} Которая, между прочим, работает за $O(N)$, но есть нюанс:)\newline 
Пример такой сортировки работающей за $O(N \log N)$: \texttt{Heap sort}.