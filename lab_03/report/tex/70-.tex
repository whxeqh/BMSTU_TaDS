\section{Ответы на вопросы}
\begin{enumerate}
	\item \textbf{ Что такое разреженная матрица и какие схемы хранения таких матриц Вы знаете?}\par
	Разреженные матрицы — это матрицы, в которых большинство элементов являются нулями. Хранение таких матриц в традиционном виде, где каждый элемент занимает место в памяти, нерационально, так как это приводит к избыточному использованию памяти для хранения большого количества нулевых значений. Для повышения эффективности используются специализированные способы хранения, такие как:
	\begin{itemize}
		\item \textbf{COO (Coordinate List)}: хранение ненулевых элементов в виде списка их значений с указанием соответствующих координат (строка, столбец, значение).
		\item \textbf{CSR (Compressed Sparse Row)}:хранение только ненулевых элементов построчно с добавлением массива индексов для указания начала каждой строки.
		\item \textbf{CSC (Compressed Sparse Column)}: аналогичная схема CSR, но используется для столбцов вместо строк.
	\end{itemize}
	\item \textbf{Каким образом и сколько памяти выделяется под хранение разреженной и обычной матрицы?}\par
	В обычной матрице каждый элемент хранится в памяти компьютера, поэтому общий размер матрицы составляет:
	\begin{equation}
		matrix\_size = rows \times columns \times \text{sizeof}(T)
		\label{eq:matrix_size}
	\end{equation}
	Где: 
	\begin{itemize} 
		\item $rows$ — количество строк 
		\item $columns$ — количество столбцов
		\item $T$ — тип данных
	\end{itemize}
	В разреженной матрице хранятся только ненулевые значения и их индексы. Размер такой матрицы складывается из размера трёх векторов. Размер вектора $A$ равен произведению количества ненулевых элементов на их тип:
	\begin{equation}
		A.size = nonzero \times \text{sizeof}(T_1)
		\label{eq:A_size}
	\end{equation}
	Где:
	\begin{itemize}
		\item $nonzero$ — количество ненулевых элементов.
		\item $T_1$ — тип данных
	\end{itemize}
	\vspace{1.5em}
	Размер вектора $IA$ также равен произведению количества ненулевых элементов на их тип:
	\begin{equation}
		IA.size = nonzero \times \text{sizeof}(T_2)
		\label{eq:IA_size}
	\end{equation}
	А размер вектора $JA$ равен произведению количества столбцов на тип:
	\begin{equation}
		JA.size = columns \times \text{sizeof}(T_3)
		\label{eq:JA_size}
	\end{equation}
	Тогда, без учёта выравнивания, весь размер будет состоять из сумм размеров векторов (\ref{eq:A_size}), (\ref{eq:IA_size}) и (\ref{eq:JA_size}):
	\begin{equation}
		csc\_matrix\_size = A.size + IA.size + JA.size
		\label{eq:csc_matrix_size}
	\end{equation}
	\item \textbf{Каков принцип обработки разреженной матрицы}
	Принцип заключается в том, что происходит обработка только ненулевых элементов, так как нулевые не повлияют на результат. За счёт этого получается улучшить производительность программы
	
	\item \textbf{В каком случае для матриц эффективнее применять стандартные алгоритмы обработки матриц? От чего это зависит?}
	Исходя из моего исследования, стандартные алгоритмы эффективнее применять при количестве ненулевых элементов матриц более чем $30\%$ от общего количества элементов. Это связано с более тяжелым доступом к элементам матрицы через ее разреженное представление 
\end{enumerate}