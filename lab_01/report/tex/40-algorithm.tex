\chapter{Описание алгоритма}
\begin{enumerate}
	\item Ввод чисел
	\begin{itemize}
		\item Проверка корректности введенных чисел
		\item Запись чисел в структуру или вывод сообщения об ошибке
	\end{itemize}
	\item Деление целого числа на вещественное
	\begin{itemize}
		\item Создается переменная ans типа bdouble\_t, которая будет возвращена из функции. Происходит вычисление порядка: умножить на -1 экспоненту делителя и сложить с разностью экспоненты делимого и длины мантиссы делителя
		\item Начинается цикл с пост условием, пока не заполнено 40 цифр в мантиссе или делимое не равно 0
		\item Определяется длина неполного делимого и частное неполного делимого и делителя
		\item Находится вычитаемое для неполного делителя, за счет умножение делимого на частное 
		\item В мантиссу записывается частное неполного делимого и делителя 
		\item Из неполного делимого вычитается произведение делимого и частного
		\item Если уже заполнено 40 цифр в мантиссе или оставшееся число равно 0, то цикл деления завершается
	\end{itemize}
	\item Вывод результата деления
	\begin{itemize}
		\item Если делитель равен нулю, выводится сообщение о невозможности деления на ноль
		\item Если экспонента в результате больше 99999, выводится сообщение о достижении машинной бесконечности
		\item  Если экспонента в результате меньше 99999, выводится сообщение о достижении машинного нуля
		\item Иначе выводится ответ в формате [+-]?(0.[0-9]*)E[+-]?([0-9]*)
	\end{itemize}
\end{enumerate}

