\chapter{Ответы на вопросы}
\noindent\textbf{1. Каков возможный диапазон чисел, представляемых в ПК?}\par
Для 64-разрядных систем максимально возможное знаковое представление целого числа находится в отрезке [-9223372036854775807; -9223372036854775806]. Максимально возможное беззнаковое представление целого числа находится в отрезке [0; 18 446 744 073 709 551 615]
Для хранения вещественных чисел в 64-разрядных системах максимально под представление мантиссы отводится 52 разряда, а под представление порядка – 11 разрядов. В этом случае возможные значения чисел находятся в диапазоне в отрезке [3.6E–4951; 1.1E+4932].\newline

\noindent\textbf{2. Какова возможная точность представления чисел, чем она определяется?}\par
Точность представления чисел выражается в количестве памяти, выделенной под их хранение. Если рассматривать числа с плавающей точкой, то обычно для float отводится 4 байт, для double 8 байт, а для long double 16 байт. Память для хранения таких чисел распределяется на хранение знака, мантиссы и экспоненты. Соответственно, чем больше памяти выделено под мантиссу и экспоненту - тем более точными будут вычисления. Если же точности не хватает, программист должен сам реализовать хранение чисел с плавающей точкой\newline

\noindent\textbf{3. Какие стандартные операции возможны над числами?}\par
Для любых чисел:сложение, вычитание, умножение, деление, сравнение
Дополнительные операции только для целых чисел: взятие остатка, побитовые сдвиги

\noindent\textbf{3. Какой тип данных может выбрать программист, если обрабатываемые
	числа превышают возможный диапазон представления чисел в ПК?}\par
Никакой из стандартных. В таких случаях программист должен написать свою структуру данных, которая будет покрывать необходимый для задачи диапазон значений.

\noindent\textbf{4. Как можно осуществить операции над числами, выходящими за
	рамки машинного представления?}\par
Необходимые операции программист должен реализовать сам, используя структуру для хранения длинного числа