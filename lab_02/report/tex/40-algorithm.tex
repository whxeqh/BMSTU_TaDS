\section{Описание алгоритма}
\begin{enumerate}
	\item Ввод пункта меню
	\item Выбран пункт 1:
	\begin{itemize}
		\item Информация о программе выводится на экран
	\end{itemize}
	\item Выбран пункт 2:
	\begin{itemize}
		\item Вводится название файла, из которого будет происходить запись
		\item Список стран из файла загружается в массив структур
	\end{itemize}
	\item Выбран пункт 3:
	\begin{itemize}
		\item Вводится название файла, в который будет происходить запись
		\item Список стран из массива структур загружается в файл
	\end{itemize}
	\item Выбран пункт 4:
	\begin{itemize}
		\item Вводится страна из консоли
		\item Страна добавляется в массив, если там есть место
	\end{itemize}
	\item Выбран пункт 5:
	\begin{itemize}
		\item Вводится страна из консоли
		\item Если страна есть в массиве - она удаляется
		\item Длина массива уменьшается, если страна была удалена
	\end{itemize}
	\item Выбран пункт 6-9:
	\begin{itemize}
		\item Список стран/ключей выводится в консоль
	\end{itemize}
	\item Выбран пункт 10-11:
	\begin{itemize}
		\item Список фильмов или список ключей сортируется по столице
	\end{itemize}
	\item Выбран пункт 12:
	\begin{itemize}
		\item Производится сортировка фильмов при разном количестве записей
		\item Выводится статистика сортировки
		\item При необходимости можно построить графики зависимости времени от количества фильмов
	\end{itemize}
\end{enumerate}

\subsection{Основные функции программы}
\begin{enumerate}
	\item \texttt{int read\_country(FILE *file\_in, country\_t *country);}
	\begin{itemize}
		\item Читает запись страны из файла
		\item \textbf{file\_in}: Входной файл
		\item \textbf{*country}: Указатель на страну для записи
		\item Возвращает код ошибки
	\end{itemize}
	\item \texttt{int add\_country\_top(country\_t countries[], const country\_t country, size\_t *length);}
	\begin{itemize}
		\item Добавляет запись страны в конец массива структур
		\item \textbf{countries[]}: Массив структур стран
		\item \textbf{country}: Страна для добавления
		\item Возвращает код ошибки
	\end{itemize}
	\item \texttt{void delete\_in\_array(country\_t *countries, size\_t *length, const size\_t pos);}
	\begin{itemize}
		\item Удаляет запись страны из массива структур
		\item \textbf{*countries}: Массив структур стран
		\item \textbf{length}: Количество элементов массива
		\item \textbf{pos}: Позиция для удаления
		\item Возвращает код ошибки
	\end{itemize}
	\item \texttt{bool find\_in\_array(country\_t *countries, const size\_t length, char *field, size\_t *pos);}
	\begin{itemize}
		\item Находит позицию страны в массиве структур по указанному полю
		\item \textbf{*countries}: Массив структур стран
		\item \textbf{length}: Количество элементов массива
		\item \textbf{*field}: Поле для сравнения
		\item \textbf{pos}: Позиция найденной страны в массиве
		\item Возвращает код ошибки
	\end{itemize}
	\item \texttt{void bubble\_sort\_countries(country\_t *countries, const int len)}
	\begin{itemize}
		\item Сортирует массив структур стран
		\item \textbf{*countries}: Массив структур стран
		\item \textbf{len}: Количество элементов массива
	\end{itemize}
	\item \texttt{void bubble\_sort\_keys(key\_t *keys, const int len)}
	\begin{itemize}
		\item Сортирует массив ключей
		\item \textbf{*countries}: Массив ключей
		\item \textbf{len}: Количество элементов массива
	\end{itemize}
\end{enumerate}
